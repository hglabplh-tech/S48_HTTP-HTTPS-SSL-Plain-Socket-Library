\documentclass[10pt,a4paper,english]{article}
\usepackage{babel}
\usepackage{ae}
\usepackage{aeguill}
\usepackage{shortvrb}
\usepackage[latin1]{inputenc}
\usepackage{tabularx}
\usepackage{longtable}
\setlength{\extrarowheight}{2pt}
\usepackage{amsmath}
\usepackage{graphicx}
\usepackage{color}
\usepackage[dvipsnames]{xcolor}
\usepackage{multirow}
\usepackage{multicol}
\usepackage{ifthen}
\usepackage[colorlinks=true,linkcolor=blue,urlcolor=blue]{hyperref}
\usepackage[DIV12]{typearea}
%% generator Docutils: http://docutils.sourceforge.net/
\newlength{\admonitionwidth}
\setlength{\admonitionwidth}{0.9\textwidth}
\newlength{\docinfowidth}
\setlength{\docinfowidth}{0.9\textwidth}
\newlength{\locallinewidth}
\newcommand{\optionlistlabel}[1]{\bf #1 \hfill}
\newenvironment{optionlist}[1]
{\begin{list}{}
  {\setlength{\labelwidth}{#1}
   \setlength{\rightmargin}{1cm}
   \setlength{\leftmargin}{\rightmargin}
   \addtolength{\leftmargin}{\labelwidth}
   \addtolength{\leftmargin}{\labelsep}
   \renewcommand{\makelabel}{\optionlistlabel}}
}{\end{list}}
\newlength{\lineblockindentation}
\setlength{\lineblockindentation}{2.5em}
\newenvironment{lineblock}[1]
{\begin{list}{}
  {\setlength{\partopsep}{\parskip}
   \addtolength{\partopsep}{\baselineskip}
   \topsep0pt\itemsep0.15\baselineskip\parsep0pt
   \leftmargin#1}
 \raggedright}
{\end{list}}
% begin: floats for footnotes tweaking.
\setlength{\floatsep}{0.5em}
\setlength{\textfloatsep}{\fill}
\addtolength{\textfloatsep}{3em}
%__________________________________________________________________________________
\newcommand{\code}[1]{{\tt{#1}}}
\renewcommand{\textfraction}{0.5}
\renewcommand{\topfraction}{0.5}
\renewcommand{\bottomfraction}{0.5}
%__________________________________________________________________________________
\setcounter{totalnumber}{50}
\setcounter{topnumber}{50}
\setcounter{bottomnumber}{50}
% end floats for footnotes
% some commands, that could be overwritten in the style file.
\newcommand{\rubric}[1]{\subsection*{~\hfill {\it #1} \hfill ~}}
\newcommand{\abbrhighcol}[1]{\textbf{\textit{#1}}}
\newcommand{\titlereference}[1]{\textsl{#1}}
% end of "some commands"
\title{An implementation of HTTP /HTTPS and SSL for Scheme 48 }
\author{Harald Glab-Plhak {\textless}\href{mailto:hglabplhak@icloud.com}{hglabplhak@icloud.com}{\textgreater}}
\date{\today{}}
%______________________________________________________________________________________
\hypersetup{
pdftitle={An implementation of HTTP /HTTPS and SSL for Scheme 48 },
pdfauthor={Harald Glab-Plhak (staatl.gepr. Inf) {\textless}hglabplhak@icloud.com{\textgreater}}%;Mike Sperber (CEO Active Group Tübingen) {\textless}michael.sperber@active-group.de{\textgreater}}
}

%______________________________________________________________________________________
\begin{document}

%______________________________________________________________________________________

%\textbf{Author}: 
%Mike Sperber. (CEO Active Group Tuebingen) {\textless}\href{mailto:michael.sperber@active-group.de}{michael.sperber@active-group.de}{\textgreater} \

%______________________________________________________________________________________

\maketitle
\tableofcontents

%______________________________________________________________________________________
\begin{multicols}{2}
\section{Abstract}

\begin{flushleft}
What we like to do is to support \abbrhighcol{SSL / HTTP  / HTTPS} in Scheme 48. The interface for that will be implemented nearly implemented as Domain Specific Language for Networking. I will use LibreSSL and (TODO Fill in) as external Libraries to avoid errors and bugs in the implementation by doing that on my own with much less time than the developers and designers of this components have due to much more manpower.  
The goal is to implement the \abbrhighcol{HTTP /1.1} and \abbrhighcol{HTTP/2} and \abbrhighcol{HTTP/3} as well. To get something  like a pseudo standard,  the interface is inspired slightly by the HTTP-EASY Library
\href{https://pkgs.racket-lang.org/package/http-easy-lib}{HTTP-EASY (Lib Site)}  which is implemented in racket which is also inspired by scheme and which also has an R6RS Scheme standard implemented. 
The implementation and interface is committed to the requirements defined by the following RFC Documents:
\end{flushleft}

\begin{flushleft}
\hspace*{1em}-- \href{https://datatracker.ietf.org/doc/html/rfc5246}{RFC-5246 (TLS V 1.2)}\\The Transport Layer Security (TLS) Protocol Version 1.2 Authors:   T. Dierks
Independent,  E. Rescorla RTFM, Inc.\\
\end{flushleft}
\begin{flushleft}
\hspace*{1em}-- \href{https://datatracker.ietf.org/doc/html/rfc8446}{RFC-8446 (TLS V 1.3)}\\The Transport Layer Security (TLS) Protocol Version 1.3 Author(s): E. Rescorla Mozilla\\
\end{flushleft}
\begin{flushleft}
\hspace*{1em}-- \href{https://datatracker.ietf.org/doc/html/rfc6101}{RFC-6101 (SSL V 3.0)}\\The Secure Socket Layer  (SSL) Protocol Version 3.0 Authors: A. Freier, P. Karlton Netscape Communications, P. Kocher Independent Consultant\\
\end{flushleft}
\begin{flushleft}
\hspace*{1em}-- \href{https://datatracker.ietf.org/doc/html/rfc3986}{RFC-3986 (URI)}\\The Uniform Resource Identifier (URI) Authors: T. Berners-Lee W3C/MIT, R. Fielding Day. Software, L. Masinter Adobe Systems\\
\end{flushleft}
\begin{flushleft}
\hspace*{1em}-- \href{https://www.rfc-editor.org/rfc/rfc9112.pdf}{RFC-9112 (HTTP/1.1)}\\The specification of the Hyper Text Transfer Protocol / 1.1 ( Authors: R. Fielding, Ed. , M. Nottingham, Ed.,  J. Reschke, Ed. ) \\
\hspace*{1em}-- \href{https://www.rfc-editor.org/rfc/rfc9110.pdf}{RFC-9110 (HTTP Semantics)} \\The semantics specification of the Hyper Text Transfer Protocol (Authors: R. Fielding, Ed., M. Nottingham, Ed., J. Reschke, Ed.)\\
\hspace*{1em}-- \href{https://www.rfc-editor.org/rfc/rfc9113.pdf}{RFC-9113 (HTTP/2)} \\The specification of the Hyper Text Transfer Protocol / 2 ( Authors:             M. Thomson, Ed. , C. Benfield, Ed. )\\
\hspace*{1em}-- \href{https://www.rfc-editor.org/rfc/rfc9114.pdf}{RFC-9114 (HTTP/3)} \\The specification of the Hyper Text Transfer Protocol / 3( Authors: M. Bishop, Ed. ) \\
\end{flushleft}\begin{flushleft}
In addition to the client library a full  functional toolkit will be implemented for developing application containers with RESTful|\footnote{\abbrhighcol {REST Requests:}The \abbrhighcol {R}epres\abbrhighcol  {E}ntational abbrhighcol{S}tate abbrhighcol{T}ransfer API is an architectural style and the RESTful services follows this specification to design a special form of HTTP requests.} interfaces (e.g. using YAML definitions) or at last my be application containers and something like OCSP\footnote{ \abbrhighcol {OCSP}: The \abbrhighcol  {O}pen \abbrhighcol {C}ertificate \abbrhighcol  {S}ervice \abbrhighcol  {P}rotocol is used to implement servers being requested using OCSP Requests and defined OCSP  Responses for checking the correctness and the validity of a certificate.} Services.
This will be really enough for the first step.
\end{flushleft}

\section{Introduction}

\begin{flushleft}
Scheme 48 is a SCHEME interpreter actually following the R5RS standard but the R6RS is nearly ready for delivery. In this implementation up to now there is a net component implementing socket communication.
The first step to reach our goal will be to introduce a SSL capable implementation using the current implementation for port and channel handling to get an appropriate layer definition for  implementing the things for HTTP / HTTPS.
\end{flushleft}
\begin{flushleft}
Short explanation of the Plain Socket / SSL Layer:
For the SSL Layer we built up a little syntax with define-syntax,  which is a define for a new generic usable function definition which contains the protocol definition, the CTX the Certificate and the connection parameters in a record after that this command executes a handshake and  open connection in a way that we  get a port number which can be used either by LibreSSL or by the native C socket API to do HTTP / HTTPS.
Short explanation of the HTTP / HTTPS Layer:
The HTTPS layer  is defined on top of the SSL Layer by the  next \code{define-syntax}.
\end{flushleft}

\rubric{The Sockets implementations }
\begin{flushleft}
First of all we have to design our interface and the syntax and attributes and functional syntactic layout of our tiny DSL. After that it is necessary to build the low level functionality in C so that we can then build our things in scheme. I prefer this way because it is good to see an early success so that the work is not too long before knowing that the concept works. The first part that we develop of course. also as a base for the DSL is a. procedural interface (function calls).
\end{flushleft}
\rubric{Issues for the project}
\begin{flushleft}
Here are some issues  which are important for our task:
\begin{itemize}
\item We have to plan that the whole implementation is threadsafe

\item The implementation uses the existing port interface of Scheme48

\item We use some kind of pseudo code to describe the logic here

\item On top the URL encoding has to be implemented as well as Basic and Certificate AUTH Headers 
\end{itemize}
\end{flushleft}

\rubric{The DSL definitions}
\begin{flushleft}
The DSL\footnote{\abbrhighcol{DSL: } A \abbrhighcol {D}omain \abbrhighcol{S}pecific \abbrhighcol{L}anguage is a language normally built up in the host language. A DSL is designed for special tasks and areas e.g.: Windowing.} for the SSL Layer itself will be designed by using the define-syntax and the other hygienic macro functionality.
\end{flushleft}
A Domain specific language has to be defined in a way that the Function / Keywords can be used inside the definition of the syntax for the DSL

\rubric{The SSL syntactic layer}
\begin{flushleft}
 At first we need a syntax to declare a Socket\abbrhighcol  {Plain/ SSL}. Here we have to think about connection parameters and about parameters and attributes for \abbrhighcol  {SSL / TLS} (e.g. \abbrhighcol  {CTX} and certificate keys and a specific Algorithm for transport. 
The\abbrhighcol  {CTX} will be a structure/ record-type as well as the definition of the base connection parameters. 
\end{flushleft}
\section{The SSL procedural interface in C / Scheme}

\begin{flushleft}
\end{flushleft}



%_________________________________________________________________________-------
\section{APPENDIX A: Terminology Explained in short}
\subsection{The HTTP 1.1 / 2 / 3 Protocol}
\begin{flushleft}
The HTTP\footnote{\abbrhighcol{HTTP:}  The \abbrhighcol{H}yper \abbrhighcol{T}ext \abbrhighcol {T}ransport \abbrhighcol {P}rotocol used for transport  e.g. \abbrhighcol{HTML} or designing  \abbrhighcol{REST Requests} . The protocol is implemented by all Web Browsers and is used all over the internet to transport / requesting and servicing data in a structured form with defined requests like \abbrhighcol{GET, PUT, DELETE, HEAD}. All of these requestshave a defined response format.} protocol has its origin in 
\end{flushleft}




\end{multicols}
\end{document}