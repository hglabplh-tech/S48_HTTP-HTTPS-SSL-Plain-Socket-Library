\documentclass[10pt,a4paper,english]{article}
\usepackage{babel}
\usepackage{ae}
\usepackage{aeguill}
\usepackage{shortvrb}
\usepackage[latin1]{inputenc}
\usepackage{tabularx}
\usepackage{longtable}
\setlength{\extrarowheight}{2pt}
\usepackage{amsmath}
\usepackage{graphicx}
\usepackage{color}
\usepackage{multirow}
\usepackage{ifthen}
\usepackage[colorlinks=true,linkcolor=blue,urlcolor=blue]{hyperref}
\usepackage[DIV12]{typearea}
%% generator Docutils: http://docutils.sourceforge.net/
\newlength{\admonitionwidth}
\setlength{\admonitionwidth}{0.9\textwidth}
\newlength{\docinfowidth}
\setlength{\docinfowidth}{0.9\textwidth}
\newlength{\locallinewidth}
\newcommand{\optionlistlabel}[1]{\bf #1 \hfill}
\newenvironment{optionlist}[1]
{\begin{list}{}
  {\setlength{\labelwidth}{#1}
   \setlength{\rightmargin}{1cm}
   \setlength{\leftmargin}{\rightmargin}
   \addtolength{\leftmargin}{\labelwidth}
   \addtolength{\leftmargin}{\labelsep}
   \renewcommand{\makelabel}{\optionlistlabel}}
}{\end{list}}
\newlength{\lineblockindentation}
\setlength{\lineblockindentation}{2.5em}
\newenvironment{lineblock}[1]
{\begin{list}{}
  {\setlength{\partopsep}{\parskip}
   \addtolength{\partopsep}{\baselineskip}
   \topsep0pt\itemsep0.15\baselineskip\parsep0pt
   \leftmargin#1}
 \raggedright}
{\end{list}}
% begin: floats for footnotes tweaking.
\setlength{\floatsep}{0.5em}
\setlength{\textfloatsep}{\fill}
\addtolength{\textfloatsep}{3em}
\renewcommand{\textfraction}{0.5}
\renewcommand{\topfraction}{0.5}
\renewcommand{\bottomfraction}{0.5}
\setcounter{totalnumber}{50}
\setcounter{topnumber}{50}
\setcounter{bottomnumber}{50}
% end floats for footnotes
% some commands, that could be overwritten in the style file.
\newcommand{\rubric}[1]{\subsection*{~\hfill {\it #1} \hfill ~}}
\newcommand{\titlereference}[1]{\textsl{#1}}
% end of "some commands"
\title{An implementation of HTTP /HTTPS and SSL for Scheme 48 }
\author{}
\date{}
\hypersetup{
pdftitle={An implementation of HTTP /HTTPS and SSL for Scheme 48 },
pdfauthor={Harald Glab-Plhak (staatl.gepr. Inf) {\textless}hglabplhak@icloud.com{\textgreater};Mike Sperber (CEO Active Group Tübingen) {\textless}michael.sperber@active-group.de{\textgreater}}
}
\raggedbottom
\begin{document}
\maketitle

%___________________________________________________________________________
\begin{center}
\begin{tabularx}{\docinfowidth}{lX}
\textbf{Author}: &
	Harald Glab-Plhak  (staatl. gepr. Inf){\textless}\href{mailto:hglabplhak@icloud.com}{hglabplhak@icloud.com}{\textgreater} \\
\textbf{Author}: &
	Mike Sperber. (CEO Active Group Tuebingen) {\textless}\href{mailto:michael.sperber@active-group.de}{michael.sperber@active-group.de}{\textgreater} \\
\textbf{Date}: &
\date{\today{}}\\
\end{tabularx}
\end{center}

\maketitle
\tableofcontents

\section{Abstract}

\begin{flushleft}
What we like to do is to support SSL / HTTP  / HTTPS in scheme 48. The interface for that will be implemented nearly implemented as Domain Specific Language for Networking. I will use LibreSSL and (TODO Fill in) as external Libraries to avoid errors and bugs in the implementation by doing that on my own with much less time than the developers and designers of this components have due to much more manpower.  
The goal is to implement the HTTP /1.1 and HTTP/2 and HTTP/3 as well. To get something  like a pseudo standard,  the interface is inspired slightly by the HTTP-EASY Library
\href{https://pkgs.racket-lang.org/package/http-easy-lib}{HTTP-EASY (Lib Site)}  which is implemented in racket which is also inspired by scheme and which also has an R6RS Scheme standard implemented. 
The implementation and interface is committed to the requirements defined by the following RFC Documents:
\end{flushleft}
\begin{flushleft}
\hspace*{1em}-- \href{https://www.rfc-editor.org/rfc/rfc9112.pdf}{RFC-9112 (HTTP/1.1)}\\The specification of the Hyper Text Transfer Protocol / 1.1 ( Authors: R. Fielding, Ed. , M. Nottingham, Ed.,  J. Reschke, Ed. ) \\
\hspace*{1em}-- \href{https://www.rfc-editor.org/rfc/rfc9110.pdf}{RFC-9110 (HTTP Semantics)} \\The semantics specification of the Hyper Text Transfer Protocol (Authors: R. Fielding, Ed., M. Nottingham, Ed., J. Reschke, Ed.)\\
\hspace*{1em}-- \href{https://www.rfc-editor.org/rfc/rfc9113.pdf}{RFC-9113 (HTTP/2)} \\The specification of the Hyper Text Transfer Protocol / 2 ( Authors:             M. Thomson, Ed. , C. Benfield, Ed. )\\
\hspace*{1em}-- \href{https://www.rfc-editor.org/rfc/rfc9114.pdf}{RFC-9114 (HTTP/3)} \\The specification of the Hyper Text Transfer Protocol / 3( Authors: M. Bishop, Ed. ) \\
\end{flushleft}\begin{flushleft}
In addition to the client library a full  functional toolkit will be implemented for developing application containers with RESTful|\footnote{REST Requests: The RepresEntational State Transfer API is an architectural style and the RESTful services follows this specification to design a special form of HTTP requests.} interfaces (e.g. using YAML definitions) or at last my be application containers and something like OCSP\footnote{OCSP: The Open Certificate Service Protocol is used to implement servers with OCSP Requests and defined OCSP  Responses to check the correctness and the validity of a certificate.} Services.
This will be really enough for the first step.
\end{flushleft}

\section{Introduction}

\begin{flushleft}
Scheme 48 is a SCHEME interpreter actually following the R5RS standard but the R6RS is nearly ready for delivery. In this implementation up to now there is a net component implementing socket communication.
The first step to reach our goal will be to introduce a SSL capable implementation using the current implementation for port and channel handling to get an appropriate layer definition for  implementing the things for HTTP / HTTPS.
\end{flushleft}
\begin{flushleft}
Short explanation of the Plain Socket / SSL Layer:
For the SSL Layer we built up a little syntax with define-syntax,  which is a define for a new generic usable function definition which contains the protocol definition, the CTX the Certificate and the connection parameters in a record after that this command executes a handshake and  open connection in a way that we  get a port number which can be used either by LibreSSL or by the native C socket API to do HTTP / HTTPS.
Short explanation of the HTTP / HTTPS Layer:
The the HTTPS layer  is defined on top of the SSL Layer by the  next define-syntax.
\end{flushleft}

\subsection{The Plain Sockets and SSL implementations }
\begin{flushleft}
First of all we have to design our interface and the syntax and attributes and functional syntactic layout of our tiny DSL. After that it is necessary to build the low level functionality in C so that we can then build our things in scheme. I prefer this way because it is good to see a early success so that the work is not to long before knowing that the concept works. 
\end{flushleft}

\subsection{The DSL Layer}
\begin{flushleft}
The DSL itself will be designed by using the define-syntax and the other hygienic macro functionality. At first we need a syntax to declare a Socket Plain/ SSL. Here we have to think about connection parameters and about parameters and attributes for SSL / TLS (e.g. CTX and certificate keys and a specific Algorithm for transport. 
The CTX will be a structure/ record-type as well as the definition of the base connection parameters. \end{flushleft}


\section{APPENDIX A: Terminology Explained in short}
\subsection{The HTTP 1.1 / 2 / 3 Protocol}
\begin{flushleft}
The HTTP\footnote{HTTP:  The Hyper Text Transport Protocol used for transport  e.g. HTML or designing REST Requests . The protocol is implemented by all Web Browsers and is used all over the internet to transport / requesting and servicing data in a structured form with defined requests like GET, PUT, DELETE, HEAD} protocol has its origin in 
\end{flushleft}
\end{document}